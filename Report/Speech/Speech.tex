\documentclass[a4paper]{article}

\usepackage[T3]{fontenc}
\usepackage[utf8]{inputenc}
\usepackage[russian]{babel}

\usepackage{fullpage} % Package to use full page
\usepackage{parskip} % Package to tweak paragraph skipping

\usepackage{mhchem}

\usepackage{geometry}
 \geometry{
 a4paper,
 total={170mm,257mm},
 left=5mm,
 top=5mm,
 right=5mm
}

\makeatletter
\newcommand*{\rom}[1]{\expandafter\@slowromancap\romannumeral #1@}
\makeatother

\begin{document}
\begin{itemize}
\item Представиться!
\item В рамках работы по проекту РНФ (Российского Научного Фонда) по созданию супрамолекулярных машин и фотоуправляемых устройств мы исследовали простейших представителей биарилиденциклопентанового ряда, в частности содержащий один незамещенный бензильный фрагмент. 
(Молекулярные машины, об которых идет речь, в частности, псевдоротаксановыые комплексы).
\item \textbf{\rom{1}} \\
В реакцию вводился свежеприготовленный енамин. Нагревание реакционной смеси проходило в колбе с насадкой Дина-Старка (с азеотропной отгонкой) в течение 20 часов. Для очистки вещества использовалась перекристаллизация из циклогексана и спирта (наилучших вариантом оказалось использование чистого циклогексана). Вещество фотолабильно поэтому в очистке ограничились пока что только перекристаллизацией. \\
Физико-химические характеристики сопоставимы с литературными. Аналитическая чистота, установленная по спектру ЯМР, примерно равна 95$\%$. 
Винильный протон в мультиплете с химическим сдвигом 7.40. \\
Мононенон -- бежевое, слегка коричневатое соединение.
\item \textbf{\rom{2}} \\
Исследовалась возможность проведения перекрестной конденсации полученного моноенона с донорными и акцепторными альдегидами. Начнем (начали) с донорного альдегида. Реакция проводилась в щелочной среде в смеси $\ce{EtOH} : \ce{H2O}$ в соотношении примерное 7:1. Через короткое время продукт выпадает в виде грязно-желтого осадка. \\ 
В рамках нашей работы мы доказали принципиальную возможность проведения перекрестной конденсации, задача оптимизации методики не стояла. \\
Физико-химические характеристики сопоставимы с литературными. Сняты спектры 1Н и 13С. \\
Из-за неполного сопряжения в системе (несмотря на то, что структура плоская) винильные протоны имеют один и тот же химический сдвиг. (Показать на тот мультиплет, в котором они сидят.) Мультиплет с химическим сдвигом 7.58.
\item \textbf{\rom{3}} \\
Затем исследовалась возможность протекания перекрестной конденсации с акцепторным альдегидом. Как и в предыдущем случае реакция проводилась в щелочной среде в смеси $\ce{EtOH} : \ce{H2O}$ в соотношении примерно 7:1. Продукт быстро выпадает в виде оранжевато-желтого осадка. \\
(Низкие выходы, причем примерно одинаковые значения выходов, возможно говорят о том, что моноенон в используемых условиях претерпевает поликонденсацию). \\
Физико-химические характеристики сопоставимы с литературными.
Здесь стоит отметить, что в протонном спектре винильные протоны разделяются: винильные протон при  пиридине идет синглетом с химическим сдвигом 7.57, а второй винильный протон идет в мультиплете с химическим сдвигом примерно 7.54.
\item \textbf{\rom{4}} \\
В качестве вывода стоит сказать, что были получены индивидуальные, хроматографически чистые вещества. Сейчас находимся в процессе получения полного набора спектральных характеристик.
Успешно синтезирован набор несимметричных биарилиденов. Впоследствие ставиться задача получить полный набор (?) несимметричных биарилиденов циклопентанонового ряда.
\item \textbf{\rom{5}} \\
В рамках курсового проекта выполнялась работа над соединениями нового класса, обладающих пара-стериловым остовом сопряжения. Стоит заметить, что публикация соединений этого класса производилась лишь однажды. 
Реакция проводилась при нагревании с насадкой Дина-Старка в течение 15 часов. В результате получается черная густая смолообразная масса. Выделение вещества из этой массы производилось флэш-хроматографией с элюентом этилацетат - петролейный эфир. Выделены ярко-желтые кристаллы продукта. Выход у нас: 34.89\%, выход в методике: 39\%. Видимо это говорит о том, что различные побочные процессы конденсации протекают при используемых условиях, это не связано с косячностью моих рук при выделении. \\ 
Физико-химические характеристики сопоставимы с представленными в той публикации. Соединение охарактеризовано по спектрам 1Н, 13С. 
Производились попытки ввести полученный диенон в реакции перекрестной конденсации с альдегидами как донорной, так и акцепторной природы, все окончились неудачно. Видимо неудачность этих попыток говорит о том, что структура устойчива. Несмотря на наличие 2 активных центров, попытки введения этого диенона в реакцию конденсации приводят либо к осмолению реакционной массы, либо к выделению исходных реагентов. 
\end{itemize}

На случай вопроса, почему работа идет с простыми моделями, сказать, что идет отработка схемы получения несимметричных диенонов.\\
Биарилидены используются как индикаторы на на металлы (добавить металл и посветить) -- фотоуправляемые машины. 
Диеноны с пара-стериловым остовом в публикации исследовались на наличие противораковой, противоопухолевой активности.
\end{document}