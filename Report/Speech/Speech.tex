\documentclass[a4paper]{article}

\usepackage[T3]{fontenc}
\usepackage[utf8]{inputenc}
\usepackage[russian]{babel}

\usepackage{fullpage} % Package to use full page
\usepackage{parskip} % Package to tweak paragraph skipping

\makeatletter
\newcommand*{\rom}[1]{\expandafter\@slowromancap\romannumeral #1@}
\makeatother

\begin{document}
В рамках работы по проекту РНФ (Российского Научного Фонда) по созданию супрамолекулярных машин и фотоуправляемых устройств мы исследовали простейших представителей биарилиденциклопентанового ряда, в частности содержащий один незамещенный бензильный фрагмент. \\
(Молекулярные машины, об которых идет речь, в частности, псевдоротаксановыые комплексы).\\
\textbf{\rom{1}} \\
Использовался свержеприготовленный енамин. Вкратце обрисовать методику. физико-химические характеристики сопоставимы с литературными. Аналитическая чистота, установленная по спектру ЯМР, примерно равна 95$\%$. Очистка вещества производилась исключительно перекристаллизационным методом из-за того, что моноенон является фотонеустойчивым соединением.
\textbf{rom{2}} \\
Исследовалась возможность проведения перекрестной конденсации с донорными и акцепторными альдегидами. Была доказана принципиальная возможность перекрестной конденсации. Низкие выходы (причем примерно одинаковые значения выходов) возомжно говорят о том, что моноенон в используемых условиях претерпевает поликонденсацию. Также стоит отметить, что задача оптимизации методики не ставилась.
ВНИМАТЕЛЬНО ПОСМОТРЕТЬ НА ПОЛОЖЕНИЯ ВИНИЛОВ ВО ВСЕХ СТРУКТУРАХ. Из-за неполного сопряжения в системе (несмотря на то, что структура плоская) винильные протоны имеют один и тот же химический сдвиг. \\
\textbf{\rom{3}} \\
Здесь исследовалась возможность конденсации с акцепторными альдегидом. ВИНИЛЬНЫЕ ПРОТОНЫ ИМЕЮТ РАЗНЫЕ ХИМИЧЕСКИЕ СДВИГИ. 
\\
В качестве вывода стоит сказать, что были получены индивидуальные, хроматографически чистые вещества. Сейчас находимся в процессе получения полного набора спектральных характеристик.
Успешно синтезирован набор несимметричных биарилиденов. Впоследствие ставиться задача получить полный набор (?) несимметричных биарилиденов циклопентанонового ряда.
\\
В рамках курсового проекта выполнялась работа над соединениями нового класса, обладающих пара-стериловым остовом сопряжения. Стоит заметить, что соединения этого класса публиковались лишь однажды. Описать вкратце процесс выделения, видимо из-за неопытности выход оказался значительно меньше выхода, указанного в методике. Соединение охарактеризовано по спектрам 1Н, 13С. Вкратце рассказать о поптыках ввести диенон в реакции перекрестной конденсации. Видимо неудачность этих попыток говорит о том, что структура устойчива. Казалось бы есть 2 активных центра, но в результате попыток ввести его в конденсацию либо идет осмоление реакционной массы, либо выделяются исходные (либо и то и другое).
\\
На случай вопроса, почему работа идет с простыми моделями, сказать, что идет отработка схемы получения несимметричных диенонов.\\
Биарилидены используются как индикаторы на на металлы (добавить металл и посветить) -- фотоуправляемые машины. 

\end{document}