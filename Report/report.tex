\documentclass[a4paper]{article}

\usepackage[T3]{fontenc}
\usepackage[utf8]{inputenc}
\usepackage[russian]{babel}

\usepackage{fullpage} % Package to use full page
\usepackage{parskip} % Package to tweak paragraph skipping

\usepackage{amsmath, amssymb}
\usepackage{graphicx}
\usepackage[colorinlistoftodos]{todonotes}

\usepackage{mhchem}

\title{Отчет.}

\date{\today}

\begin{document}
\maketitle

\begin{abstract}
Abstract.
\end{abstract}

\section*{Синтез 2-бензолиденциклопентанона.}
11.45г (74.84 ммоль) N-циклопентанилморфолина, 6.61г (62.36 ммоль) свежеперегнанного бензальдегида и 60 мл бензола помещают в круглодонную колбу и нагревают с насадкой Дина-Старка в течение 20 часов. За ходом реакции следят при помощи ТСХ (элюент -- петролейный эфир : этилацетат, 4 : 1).
Затем раствор охлаждают до комнатной температуры и при перемешивании добавляют 43.5 мл 6М $\ce{HCl}$. После перемешивания в течение 2 часов органический слой отделяют и промывают водой до нейтрального pH, оставляют сушиться над $\ce{Na2SO4}$ на ночь. Затем смесь фильтруют и отгоняют бензол на роторном растворителе. Остаток охлаждают и кристаллизуют. Очистку производят перекристаллизацией из смеси этанол - циклогексан.

\section*{Синтез 2-бензолиден-5-(4-метоксибензолиден)циклопентанона.}



\end{document}